\documentclass[a4paper]{article}
\usepackage[warn]{mathtext}
\usepackage[utf8]{inputenc}
\usepackage[T2A]{fontenc}
\usepackage[english,russian]{babel}
\usepackage{multicol}
\usepackage{fancyhdr}
\usepackage{graphicx}
\usepackage{microtype}
\usepackage{wrapfig}
\usepackage{amsmath}
\usepackage{floatflt}
\usepackage{geometry} \geometry{verbose,a4paper,tmargin=2cm,bmargin=2cm,lmargin=1.5cm,rmargin=1.5cm}
\usepackage{float}
\usepackage{amssymb}
\usepackage{caption}
\usepackage{epsfig}
\usepackage{newunicodechar}

\begin{document}

\begin{titlepage}
	\centering
	\vspace{5cm}
    {\scshape\LARGE Московский физико-технический институт\par}
    

	\vspace{8cm}
	{\scshape\Large Ответы на контрольные вопросы по лабораторной работе по вакуумной электронике  \par}
	\vspace{2cm}
    {\huge\bfseries  Атомно-силовой микроскоп  \par}
	\vspace{4cm}
	\vfill
\begin{flushright}
	{\large выполнили студенты Б04-852 группы ФЭФМ}\par
	\vspace{0.3cm}
    {\LARGE Бабичек Илья}\\
    \vspace{0.1cm}
    {\LARGE Водзяновский Яромир}\\
    \vspace{0.1cm}
	{\LARGE Груздева Юлия}\\
    \vspace{0.1cm}
    {\LARGE Демидов Станислав}\\
    \vspace{0.1cm}
	{\LARGE Денисов Дмитрий}\\
\end{flushright}
	
	\vfill
Долгопрудный, 2020
% Bottom of the page
\end{titlepage}

\begin{enumerate}
    
    \item \textbf{Что такое пьезоэлектрический эффект (прямой и обратный)?} \par 
    
    \textit{Прямой пьезоэлектрический эффект} - явление возникновения электрической поляризации в кристалле под действием механического напряжения \par
    \textit{Обратный пьезоэлектрический эффект} - возникновение механических напряжений в кристалле под действием электрического поля


    \item \textbf{Чем он отличается от электрострикции?} \par 

    \textit{Электросткрикция} - это деформация диэлектрических материалов в электричкеском поле, пропорциональная квадрату напряженности электрического поля,
    в изотропных диэлектриках описывается формулой:

    \begin{equation}
        \Delta V / V = q E^2
    \end{equation}

    где $\Delta V/ V$ - относительная объемная деформация, $E$ - напряженность электрического поля, $q$ - постоянная электрострикции,
    зависящая от сжимаемости, плотности и диэлектрической приницаемости.

    Для анизотропных кристаллов электрострикцию можно описать зависимостью между 2 тензорами 2-го ранга — тензором квадрата напряжённости электрического поля и тензором деформации:

    \begin{equation}
        r_{ij} = \sum\limits_m \sum\limits_n R_{ijmn} E_m E_n
    \end{equation}

    здесь $r_{ij}$ - компонента тензора деформации, $E_m, E_n$ - составляющие электрического поля, $R_{ij}$ - коэффициент электрострикции. Число независимых коэфициентов зависит от симметрии кристаллов. 

    \begin{itemize}
        \item Электрострикция обусловлена поляризацией диэлектриков в электрическом поле, то есть смещением под действием 
        внешнего электрического поля атомов, несущих на себе электрические заряды (ионы, электрические диполи), или изменением ореинтации диполей.
        Она наблюдается у всех диэлектриков.

        \item В отличие от электрострикции прямой пьезоэлектрический эффект наблюдается только в кристаллах (твердых телах с трехмерно-периодической укладкой)
        без центра симметрии. Это объясняется тем, что при деформации кристалла центр симметрии сохраняется, а при наличии центра симметрии не может быть поляризации.

        \item Эффект электрострикции является квадратичным (четным) ф-ла (1), в то время как обратный пьезоэлектрический эффект — линейным (нечетным).
        Вследствие этого под действием переменного электрического поля частотой $\omega$ в случае электрострикции диэлектрик колеблется с удвоенной частотой $2 \omega$, 
        а в случае обратного пьезоэлектрического эффекта диэлектрик будет колебаться с той же частотой.
        
        \item В случае электрострикции направление деформации тела не зависит от направления электрического поля. В случае же пьезоэлектрического эффекта деформация прямо пропорциональна 
        напряжённости поля, причем с изменением направления поля на противоположное знак деформации меняется. 

        \item В отличие от пьезоэффекта электрострикция является необратимым эффектом.


    \end{itemize}


    \item \textbf{Запишите уравнение пьезоэлектрического эффекта. Опишите каждый тензор в уравнении. Запишите границы применимости вашего уравнения.} \par 
    
    Электрическая поляризация $P = D - \varepsilon_0 E$ связана с поверхностным зарядом, в \textit{первом приближении} увеличивается линейно относительно механического напряжения $\sigma$:

    \begin{equation}
        D = P + \varepsilon_0 E = d \sigma
    \end{equation}

    где $D$ и $E$ - векторы индукции и напряженности электрического поля, а механическое напряжение $\sigma$ и деформация $\varepsilon$ - тензоры второго ранга. 
    Пьезоэлектрический коэффициент $d$ - тензор третего ранга. В компонентном виде получаем соотношение для прямого пьезоэффекта:

    \begin{equation}
        D = \left( \begin{array}{c}
            D_1 \\
            D_2\\
            D_3 
        \end{array} \right) = 
        \left ( \begin{array}{cccccc} 
            d_{11} & d_{12} & d_{13} & d_{14} & d_{15} & d_{16} \\
            d_{21} & d_{22} & d_{23} & d_{24} & d_{25} & d_{26} \\
            d_{31} & d_{32} & d_{33} & d_{34} & d_{35} & d_{36} \\
        \end{array} \right )
        \left ( \begin{array}{c} 
            \sigma_1 \\
            \sigma_2 \\
            \sigma_3 \\
            \sigma_4 \\
            \sigma_5 \\
            \sigma_6 \\
        \end{array} \right )
    \end{equation}

    Индексы 1,2,3 относятся к кристаллическим параметрам и их можно совместить с координатными направлениями $x,y,z$ при соответсвующей орейнтации. \par
    
    Обратный пьезоэлектрический эффет дает соотношение между напряженностю электрического поля $E$ и механической деформацией $\varepsilon$:

    \begin{equation}
        \left ( \begin{array}{c} 
            \varepsilon_1 \\
            \varepsilon_2 \\
            \varepsilon_3 \\
            \varepsilon_4 \\
            \varepsilon_5 \\
            \varepsilon_6 \\
        \end{array} \right ) = 
        \left ( \begin{array}{cccccc} 
            d_{11} & d_{12} & d_{13} & d_{14} & d_{15} & d_{16} \\
            d_{21} & d_{22} & d_{23} & d_{24} & d_{25} & d_{26} \\
            d_{31} & d_{32} & d_{33} & d_{34} & d_{35} & d_{36} \\
        \end{array} \right )^T
        \left( \begin{array}{c}
            E_1 \\
            E_2\\
            E_3 
        \end{array} \right)
    \end{equation}

    Коэффициенты $d_{ij}$ тождественны прямому пьезоэлектрическому эффекту. 


    \item \textbf{В каких средах нельзя наблюдать пьезоэлектрический эффект? Покажите это математически на основе выписанного ранее уравнения.
    (Пример: кристаллический кварц $SiO_2$ применяют в радиотехнике в качестве резонатора с высокой добротностью, т.е. он точно обладает обратным пьезоэффектом,
    а кварцевое стекло с идентичным составом не обладает - почему?) Сделайте вывод.} \par 

    В зависимости от кристаллической структуры, некоторые пьезоэлектрические коэффициенты станут нулевыми или их можно приравнять к друг другу. Фактически
    вид тензора определяется кристаллическим классом, к которому материал принадлежит. Кристаллы обладающие центральной симметрией (как кремний) или изотропные 
    материалы не проявляют пьезоэлектрический эффект. \par 
    
    У кристаллов вдоль направлений симметрии зануляются пьезоэлектрические компоненты, это накладывает ограничения на направления возможной поляризации. Соответсвенно, если 
    кристалл будет обладать центральной симметрией будет иметь нулевой пьезожлектрический тензор и не будет обладать пьезоэффектом.

    Кварцевое стекло - аморфное вещество, у него нельзя выделить четкое направление поляризации ввиду отсутвия кристаллической структуры в отличе от кристаллического кварца.

    \item \textbf{Что такое керамика (как структура твердого тела)? Каким принципиальным свойством должно обладать вещество, чтобы из него можно было изготовить 
    пьезокерамику? Почему в пьезосканерах используют пьезокерамику, а не кристаллический пьезоэлектрик?} \par 

    \textit{Керамика} - неметаллический поликристаллический (агрегат кристаллов какого-либо вещества) материал, обычно получаемый спеканием порошков.\par 

    \textit{Пьезокерамика} - искусственный материал, обладающий пьезоэлектрическими и сегнетоэлектрическими свойствами, имеющий поликристаллическую структуру.
    По своей сути пьезокерамика это спрессованные зерна кристаллов (минимальный объем монокристалла, окруженный высокодефектными границами), которые \textit{должны 
    обладать пьезоэлектрическим свойством}. По химическому составу пьезокерамика сложный оксид. Большинство составов пьезокерамики основано на химических 
    соединениях с кристаллической структурой типа перовскита с формулой $ABO_3$, например $BaTiO_3$.\par

    Пьезокерамике можно придать любую форму, в частности, в сканере АСМ она имеет вид трубок, такую форму невозможно придать монокристаллу. Однако пьезокерамика обладает рядом недостатков,
    таких как нелинейность, гитсерезис, ползучесть, старение и температурный дрейф.

\end{enumerate}



	



\end{document}